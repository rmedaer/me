%%%%%%%%%%%%%%%%%%%%%%%%%%%%%%%%%%%%%%%%%%%%%%%%%%%%%%%%%%%%%%%%%%%%%%%%%%%%%%%%%%%%%%%
% Raphael Medaer Resume/CV
% XeLaTeX Template
% Version 1.0~rme (2017)
%
% This template has been downloaded from:
% http://www.LaTeXTemplates.com
%
% Original author:
% Adrien Friggeri (adrien@friggeri.net)
% https://github.com/afriggeri/CV
%
% Modified by:
% Raphael Medaer
%
% License:
% CC BY-NC-SA 3.0 (http://creativecommons.org/licenses/by-nc-sa/3.0/)
%
% Important notes:
% This template needs to be compiled with XeLaTeX and the bibliography, if used,
% needs to be compiled with biber rather than bibtex.
%
% Not important notes:
% If you're reading this, you might be interested by resume... nice !!
% Feel free to contact me ;-)
%
%%%%%%%%%%%%%%%%%%%%%%%%%%%%%%%%%%%%%%%%%%%%%%%%%%%%%%%%%%%%%%%%%%%%%%%%%%%%%%%%%%%%%%%

\documentclass[]{friggeri-cv} % Add 'print' as an option into the square bracket to remove colors from this template for printing

\usepackage{color,soul}
\setulcolor{underliner}

\hypersetup{
	pdfauthor={Raphael Medaer},
	pdftitle={Raphael Medaer - Curriculum Vitae},
	pdfsubject={Curriculum Vitae},
	pdfkeywords={rmedaer,cv,resume},
	pdfproducer={XeLateX},
	pdfcreator={XeLateX}}

\begin{document}


\input{personal.tex}

\header{Raphael}{Medaer}{Telecom \& industrial software engineer}

%----------------------------------------------------------------------------------------
%	SIDEBAR SECTION
%----------------------------------------------------------------------------------------

\begin{aside} % In the aside, each new line forces a line break
\section{contact}
\myAddress
~
\myPhoneNumber
~
\href{mailto:\myPrivateMail}{\ul{\myPrivateMail}}
\href{http://github.com/rmedaer}{\ul{gh://rmedaer}}
\section{languages}
french mother tongue
english fluency
\section{programming}
{\color{red} $\varheartsuit$} C/C++
Python, PHP, Javascript,
Java, ASM, (.*)sh
\end{aside}

%----------------------------------------------------------------------------------------
%	EDUCATION SECTION
%----------------------------------------------------------------------------------------

\section{education}

\begin{entrylist}
%------------------------------------------------
\entry
{2010--2014}
{Bachelor {\normalfont of Computer Science}}
{Ecole Supérieur d'Informatique (HEB)}
{Obtained magna cum laude with a degree in indrustrial computer sciences.}
%------------------------------------------------
\entry
{2008--2010}
{{\normalfont History of Art}}
{Université Libre de Bruxelles}
{Specialization in Musicology}
%------------------------------------------------
\end{entrylist}

%----------------------------------------------------------------------------------------
%	WORK EXPERIENCE SECTION
%----------------------------------------------------------------------------------------

\section{experience}

\begin{entrylist}
%------------------------------------------------
\entry
{2014--Now}
{\href{http://escaux.com}{\ul{Escaux SA}}}
{Wavre, Belgium}
{\emph{Software Development Engineer} \\
Developed unified communications systems and modern WebRTC softphone connected with fixed-mobile unification; these include low level and high level languages and technologies. I developed also a suite of automation and deployment tools based on OpenSource solutions. This experience brought me the following knowledges:
\begin{itemize}
\item RTC developments, architectures and protocols (e.g. SIP)
\item Network security skills, specially in Web 2.0 technologies (WebRTC, OAuth, ...)
\item Contributions to major softwares and projects: Asterisk, \href{http://lists.mindrot.org/pipermail/openssh-unix-dev/2016-December/035563.html}{\ul{OpenSSH}}, \href{https://github.com/rmedaer/mod_psm}{\ul{Apache2}}, \href{https://github.com/smira/aptly/pull/390}{\ul{Aptly}}, Debian, Ansible, ...
\item Automation and deployment models. Architecture analysis and design.
\item Mobile developments (Android).
\end{itemize}}
%------------------------------------------------
\entry
{2014}
{\href{http://www.cenaero.be/}{\ul{Cenaero}}}
{Gosselies, Belgium}
{\emph{Internship} \\
Built an advanced monitoring system for \href{https://www.top500.org/system/178439}{\ul{biggest Belgian supercomputer}}. \\
I worked with HPC technologies such as PBS, MPI, Ganglia or memcached servers.}
%------------------------------------------------
\end{entrylist}

%----------------------------------------------------------------------------------------
%	HACKER SKILLS SECTION
%----------------------------------------------------------------------------------------

\section{knowledge \& skills}

\textbf{communication:} Team playing, Agile (Scrum), DevOps, Community-Driven development \\
\textbf{unix \& linux:} Debian tools, Kernel, PAM \& NSS module developments \\
\textbf{network:} common layers, protocols++ (e.g. SIP, HTTP, SSH, ...) \\
\textbf{security:} PKI (OpenSSL, GPG), DTLS, SRTP \\ 
\textbf{electronics:} PIC, I2C, SPI, Step 7, Modbus, IPMI ... \\
\textbf{databases:} {My,Postgre}SQL, Mongo, Redis \\
 

%----------------------------------------------------------------------------------------
%	INTERESTS SECTION
%----------------------------------------------------------------------------------------

\section{interests}

\textbf{professional:} low-level technologies, high-performance computing, security sciences, events \& meetups, open-source\footnote{Resume \href{https://github.com/rmedaer/me}{\ul{license and sources on Github}}} softwares, after-works with colleagues \\
\textbf{personal:} electronics, white hacking, organizing \href{http://www.relaispourlavie.be/relays/braine-lalleud-2017#block-views-stkrfl-contacts-main-block-1}{\ul{Relay 4 life}}, music (piano), \href{https://www.lairdubois.fr/pas-a-pas/268-meuble-tv-chester-2.html}{\ul{wood working}}

\end{document}
